\documentclass[12pt]{article}

\usepackage{graphicx,url}

\usepackage[brazilian]{babel}
\usepackage[utf8x]{inputenc}
\usepackage{amssymb,amsmath}
\usepackage[T1]{fontenc}
\usepackage{verbatim}
\usepackage{hyperref}
\usepackage{fullpage}
\usepackage{multicol}
\usepackage{lmodern}
\usepackage{xcolor}
\usepackage{listings}
\usepackage[colorinlistoftodos]{todonotes}
\usepackage{fixltx2e}
\usepackage{longtable}
\usepackage{float}
\usepackage{wrapfig}
\usepackage{soul}
\usepackage{textcomp}
\usepackage{marvosym}
\usepackage{wasysym}
\usepackage{latexsym}
\usepackage{amssymb}
\usepackage{hyperref}
\tolerance=1000
\usepackage{geometry}
\geometry{left=0.7in,right=0.7in,top=1in,bottom=1in}
\providecommand{\alert}[1]{\textbf{#1}}

\lstset{%
 basicstyle=\small\ttfamily\color{black!85},
 breaklines = true,
 keywordstyle=\bfseries\color{black},
 emphstyle=\color{blue},
 columns=fullflexible,
 showstringspaces=false
}%

\lstdefinelanguage{GODSmalltalk}
  {keywords={SSSpreadsheetData, SSSheet, SSRow, SSCell},
  morestring=[b]{'},
  stringstyle=\color{purple},
  alsoletter={:}, 
  emph={new}
}

% Python environment
\lstnewenvironment{godCode}[1][]
{
\lstset{language=GODSmalltalk,
    #1}
}
{}


\graphicspath { {figures/} }
\setlength{\parindent}{0cm}

\begin{document}

\section{GODAcademics '15}

\textbf{Grupo:}\textit{Pedro Bruel, António Martins Miranda, António Castro Júnior}

\textbf{Contato:}\{pedro.bruel, amartmiranda, to.junior.25\}@gmail.com\\

GOD Academics é um agregador de informações acadêmicas. A partir de um perfil 
do Google Scholar, ou de um currículo Lattes, esta aplicação constrói 
um relatório resumindo as informações obtidas de diversas fontes.

\subsection{Objetivos}

Os objetivos para o primeiro semestre de 2015 foram:

\begin{itemize}
    \item Refatoração de Testes;
    \item Aumentar a Cobertura de Testes;
    \item Aprimorar a página \emph{web} do projeto;
    \item Implementar a busca \emph{fuzzy} de texto (distância de Levenshtein);
    \item Implementar uma interface com o Sistema Lattes.
\end{itemize}

\subsection{Refatorações e Cobertura de Testes}

Adicionamos mais casos de teste para a busca por \emph{strings}
relativas a conferências e \emph{journals}. Separamos alguns testes
de unidade que agrupavam vários métodos no mesmo caso de teste.

\subsection{Página do Projeto}

Foi implementada uma nova interface para a página \emph{web}
do GODAcademics. A \emph{Issue} 450 no \emph{redmine} descreve
alguns pedidos de alterações feitos ao grupo \emph{GODWeb},
para que nossa implementação pudesse ser utilizada.

Habilitamos as visualizações para as buscas no \emph{Google Scholar}
e currículo \emph{Lattes}. A busca agora é feita através de um único
campo, e a distinção entre as fontes é feita baseando-se na assinatura
do \emph{link} submetido.

\subsection{Busca \emph{Fuzzy}}

Foi implementado um algoritmo para a busca aproximada de texto.
O algoritmo utilizado foi o cálculo da distância de \emph{Levenshtein},
que permite calcular distâncias entre \emph{strings} e fazer a busca
\emph{fuzzy}. Por exemplo, a busca por "\emph{plos comp bio}" deve retornar
resultados para o \emph{journal} \emph{PLOS Computational Biology}.

O método implementado foi \texttt{levenshteinDistanceBetween:and:},
que recebe duas \emph{strings} e calcula a distância entre elas.

Tivemos alguns problemas com a implementação dos testes para esse algoritmo,
pois o arcabouço de testes para \emph{Smalltalk} utilizado impõe limites
para a duração dos testes de unidade.

\subsection{Interface com o Sistema Lattes}

Tivemos problemas com o interfaceamento e obtenção de informações
de Currículos \emph{Lattes}, pois o \emph{site} adotou, recentemente,
um sistema de \emph{CAPTCHA}. Uma tentativa de contornar esse problema
foi utilizar o \emph{scriptLattes}, uma ferramenta para obtenção desses dados.
No entanto, a ferramenta também não funcionou por conta do \emph{CAPTCHA}.

Decidimos então implementar o \emph{parsing} de arquivos html correspondentes
a perfis do Currículo Lattes. Desta forma, quando for possível contornar as
restrições impostas pelo sistema, já teremos a estrutura para obtenção de 
informações sobre os pesquisadores pronta.

\end{document}
